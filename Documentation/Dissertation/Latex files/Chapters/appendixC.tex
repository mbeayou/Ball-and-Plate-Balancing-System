
In this section, the angular mapping method is utilized to establish the relationship between the input angles and the corresponding output angles of the servomotors. The measurements were obtained using a smartphone, and the linear gains for the mapping were determined as \(k_x = 0.0445\) and \(_y = 0.0377\) by the following matlab code:
\begin{lstlisting}[language=Matlab , caption= matlab code for polynomial fits ]
input = [0,5,10,15,20,25,30,
35,40,45,50,55,60,65,70,75,
80,85,90,95,100,105,110,115,
120,125,130,135,140,145,150,
155,160,165,170,175,180];
output = [-7.9,-7.8,-7.6,-7.2,
-6.9,-6.4,-6,-5.1,-4.6,-3.8,-3,-2.1,-1.4,-0.5, 0.1,0.1,0.5,0.8,1.3,2,2.8,3.6,4.4,5,5.7,6.4,
7,7.6,8.1,9,9.6,9.8,9.9,10.1,10.3,10.4,10.5];
degree = 1;
% polyfit(input,output,degree);
gx = output/input
inputy = [0,5,10,15,20,25,30,35,40,45,
50,55,60,65,70,75,80,85,90,95,100,
105,110,115,120,125,130,135,140,145,150];
outputy=  [7.3,7,6.8,6.6,6.2,5.8,5.3,
4.6,4,3.4,2.7,2.1,1.5,0.7,0.2,0,
-0.9,-1.7,-2.3,-2.9,-3.8,-4.6,-5.3,
-5.8,-6.5,-7,-7.7,-8,-8.6,-9,-9.1];
gy = outputy/inputy
\end{lstlisting}

{\large \textbf{Mapping Values}}

The following table presents the measured input and output angles along with the calculated gains.
\newcolumntype{P}[1]{>{\centering\arraybackslash}p{#1}}
\begin{table}[h]
    \centering
    \begin{tabular}{|P{0.25\linewidth}| P{0.25\linewidth} | P{0.25\linewidth}|}
        \hline
        \textbf{Servomotor Angle (degrees)} & \textbf{Inclination Angle X (degrees)} & \textbf{Inclination  Angle Y (degrees)} \\
        \hline
        0    & -7.9 & 7.3   \\
        5    & -7.8 & 7.0   \\
        10   & -7.6 & 6.8  \\
        15   & -7.2 & 6.6  \\
        20   & -6.9 & 6.2  \\
        25   & -6.4 & 5.8  \\
        30   & -6.0 & 5.3  \\
        35   & -5.1 & 4.6  \\
        40   & -4.6 & 4.0  \\
        45   & -3.8 & 3.4  \\
        50   & -3.0 & 2.7  \\
        55   & -2.1 & 2.1  \\
        60   & -1.4 & 1.5  \\
        65   & -0.5 & 0.7  \\
        70   & -0.1 & 0.2  \\
        75   & 0.1  & 0.0  \\
        80   & 0.5  & -0.9 \\
        85   & 0.8  & -1.7 \\
        90   & 1.3  & -2.3 \\
        95   & 2.0  & -2.9 \\
        100  & 2.8  & -3.8 \\
        105  & 3.6  & -4.6 \\
        110  & 4.4  & -5.3 \\
        115  & 5.0  & -5.8 \\
        120  & 5.7  & -6.5 \\
        125  & 6.4  & -7.0 \\
        130  & 7.0  & -7.7 \\
        135  & 7.6  & -8.0 \\
        140  & 8.1  & -8.6 \\
        145  & 9.0  & -9.0 \\
        150  & 9.6  & -9.1 \\
        155  & 9.8  &  \\
        160  & 9.9  &   \\
        165  & 10.1 &   \\
        170  & 10.3 &   \\
        175  & 10.4 &   \\
        180  & 10.5 &   \\
        \hline
    \end{tabular}
    \caption{Mapping values for input and output angles to calculate the gains.}
    \label{tab:mapping_values}
\end{table}

